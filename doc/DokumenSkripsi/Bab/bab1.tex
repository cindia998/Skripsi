%versi 2 (8-10-2016) 
\chapter{Pendahuluan}
\label{chap:intro}
   
\section{Latar Belakang}
\label{sec:label}

Bagian ini akan diisi dengan apa yang melatarbelakangi pembuatan template skripsi ini.
Termasuk juga masalah-masalah yang akan dihadapi untuk membuatnya, termasuk kurangnya kemampuan penguasaan \LaTeX{} sehingga template ini dibuat dengan mengandalkan berbagai contoh yang tersebar di dunia maya, yang digabung-gabung menjadi satu jua.
Bagian lain juga akan dilengkapi, untuk sementara diisi dengan lorem ipsum versi bahasa inggris.

\textit{Big data} selalu datang dan terkumpul secara terus menerus. Mengklasifikasikan data merupakan salah satu proses yang diperlukan untuk mendapatkan informasi dan prediksi dari data-data yang ada. Model klasifikasi data yang dilakukan harus diperbarui seiring dengan bertambahnya data agar tetap mendapatkan informasi dan prediksi yang diinginkan.

Untuk mendapatkan hasil prediksi dengan akurasi yang tinggi, digunakan teknik yang namanya Ensemble Method. \textit{Ensemble method} merupakan teknik untuk meningkatkan akurasi pada teknik klasifikasi. Pada teknik ini, \textit{dataset training} akan dipecah secara acak, tiap bagian digunakan untuk membuat model klasifikasi. \textit{Dataset training} dipecah menggunakan teknik \textit{bagging}. \textit{Bagging} (\textit{bootstrap aggregating}) merupakan teknik untuk membagi-bagi dataset dengan menerapkan ‘\textit{sampling with replacement}’ yang dapat dimanfaatkan pada \textit{ensemble method}. Pada tahap prediksi, kasus baru akan diumpankan ke semua model. Hasil prediksi kelas akan ditentukan dari kelas dengan jumlah voting terbanyak.

Perangkat lunak akan dibuat dengan bantuan \textit{Apache Spark}. \textit{Apache Spark} adalah teknologi komputasi \textit{cluster} yang dirancang untuk komputasi cepat yang berdasar pada \textit{Hadoop MapReduce}  dan model \textit{MapReduce}. Library klasifikasi paralel untuk klasifikasi \textit{big data} pada sistem tersebar \textit{Spark} sudah tersedia pada Spark Machine Learning Library (Spark MLLib). Perlu dikembangkan agar menangani pembuatan model klasifikasi secara inkremental.

Pada skripsi ini, akan dibuat sebuah program yang mengklasifikasikan big data secara inkremental dengan menggunakan \textit{ensemble method classifier} dengan mengadopsi \textit{bagging}. Inkremental maksudnya adalah model klasifikasi data harus diperbarui seiring dengan bertambahnya data agar model tersebut dapat tetap valid, merepresentasikan \textit{big data} baru dan memiliki akurasi yang diharapkan.

\section{Rumusan Masalah}
\label{sec:rumusan}
Bagian ini akan diisi dengan penajaman dari masalah-masalah yang sudah diidentifikasi di bagian sebelumnya. 
Masalah-masalah yang ingin diselesaikan dalam skripsi ini adalah sebagai berikut.
\begin{enumerate}
\item Bagaimana cara mengimplementasikan \textit{ensemble method classifier} dengan \textit{bagging} dan voting ke dalam algoritma klasifikasi?
\item Apa saja algoritma klasifikasi yang ada pada \textit{Spark}?
\item Bagaimana cara algoritma klasifikasi yang ada pada Spark bekerja?
\item Bagaimana \textit{ensemble method} bekerja?
\item Bagaimana teknik \textit{bagging} bekerja?\\
\item Bagaimana mengimplementasikan algoritma klasifikasi pada spark dengan bahasa scala?
\end{enumerate}


\section{Tujuan}
\label{sec:tujuan}
Akan dipaparkan secara lebih terperinci dan tersturkur apa yang menjadi tujuan pembuatan template skripsi ini
Tujuan dari pembuatan perangkat lunak adalah sebagai berikut.
\begin{enumerate}
\item Mengembangkan algoritma klasifikasi pada \textit{Spark} menjadi \textit{ensemble method classifier} dengan \textit{bagging}.
\item Mempelajari jenis-jenis algoritma klasifikasi pada \textit{Spark}.
\item Mempelajari bagaimana Spark bekerja.
\item Mempelajari \textit{ensemble method}.
\item Mempelajari teknik \textit{bagging}.
\item Mempelajari bahasa Scala.
\end{enumerate}


\section{Batasan Masalah}
\label{sec:batasan}
Untuk mempermudah pembuatan template ini, tentu ada hal-hal yang harus dibatasi, misalnya saja bahwa template ini bukan berupa style \LaTeX{} pada umumnya (dengan alasannya karena belum mampu jika diminta membuat seperti itu)

\dtext{8}

\section{Metodologi}
\label{sec:metlit}
Tentunya akan diisi dengan metodologi yang serius sehingga templatenya terkesan lebih serius.

\dtext{9}

\section{Sistematika Pembahasan}
\label{sec:sispem}
Rencananya Bab 2 akan berisi petunjuk penggunaan template dan dasar-dasar \LaTeX.
Mungkin bab 3,4,5 dapt diisi oleh ketiga jurusan, misalnya peraturan dasar skripsi atau pedoman penulisan, tentu jika berkenan.
Bab 6 akan diisi dengan kesimpulan, bahwa membuat template ini ternyata sungguh menghabiskan banyak waktu.

Bab 2 akan menjelaskan tentang Big Data, Spark, algoritma klasifikasi yang ada pada Spark MLLib, bahasa pemrograman Scala.
Bab 3 akan berisi hasil dari eksplorasi di bab 2.
Bab 4 akan berisi perancangan model algoritma klasifikasi yang menggunakan ensemble method.
 